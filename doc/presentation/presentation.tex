\documentclass[dvipdfmx,13pt,aspectratio=169]{beamer}

\usepackage{animate}
% \usepackage{svg}

\usepackage{adjustbox}      % \begin{adjustbox}
\usepackage[ruled, vlined]{algorithm2e}    % \begin{algorithm2e}
\usepackage{amsmath}        % \begin{align*}
\usepackage{amssymb}        % \mathbb{A}
\usepackage{amsthm}         % \newtheorem
\usepackage{bm,bbm}         % \bm{A}, \bbm{1}
\usepackage{booktabs}       % \toprule, \midrule, \bottomrule
\usepackage{enumitem}       % \begin{enumerate}[label=(\alph*)]
\usepackage{hyperref}       % \href{URL}{text}
\usepackage{ifthen}         % \ifthenelse
\usepackage{lipsum}         % \lipsum
\usepackage{makecell}       % \makecell{L1\L2}
\usepackage{mathrsfs}       % \mathscr{A}
\usepackage{mathtools}      % \mathrlap
\usepackage{multirow}       % \multirow
\usepackage{optidef}        % \begin{mini*}{x}{f(x)}{}{}
\usepackage{orcidlink}      % \orcidlink
\usepackage{physics}        % \qty, \norm, \abs
\usepackage{subcaption}     % \captionsetup
\usepackage{subfiles}       % \subfile{file}
\usepackage{thm-restate}    % \begin{restatable}{theorem}{thm}
\usepackage{tikz}           % \begin{tikzpicture}
\usepackage{xparse}         % \NewDocumentCommand

\hypersetup{colorlinks=true,linkcolor=blue,citecolor=blue,urlcolor=blue}

\definecolor{cA}{HTML}{0072BD}
\definecolor{cB}{HTML}{EDB120}
\definecolor{cC}{HTML}{77AC30}
\definecolor{cD}{HTML}{D95319}

\newcommand{\red}[1]{\textcolor{red}{#1}}
\newcommand{\blue}[1]{\textcolor{blue}{#1}}
\newcommand{\cyan}[1]{\textcolor{cyan}{#1}}
\newcommand{\gray}[1]{\textcolor{gray}{#1}}
\newcommand{\green}[1]{\textcolor{green}{#1}}
\newcommand{\brown}[1]{\textcolor{brown}{#1}}
\newcommand{\black}[1]{\textcolor{black}{#1}}
\newcommand{\orange}[1]{\textcolor{orange}{#1}}
\newcommand{\purple}[1]{\textcolor{purple}{#1}}
\newcommand{\tccA}[1]{\textcolor{cA}{#1}}
\newcommand{\tccB}[1]{\textcolor{cB}{#1}}
\newcommand{\tccC}[1]{\textcolor{cC}{#1}}
\newcommand{\tccD}[1]{\textcolor{cD}{#1}}
\newcommand{\st}{\text{ s.t. }}
\newcommand{\Img}[1]{\mathrm{Im}\qty(#1)}
\newcommand{\Ker}[1]{\mathrm{Ker}\qty(#1)}
\newcommand{\Supp}[1]{\mathrm{supp}\qty(#1)}
\newcommand{\Rank}[1]{\mathrm{rank}\qty(#1)}
\newcommand{\floor}[1]{\left\lfloor #1 \right\rfloor}
\newcommand{\ceil}[1]{\left\lceil #1 \right\rceil}
% C++ (https://tex.stackexchange.com/questions/4302/prettiest-way-to-typeset-c-cplusplus)
\newcommand{\Cpp}{C\nolinebreak[4]\hspace{-.05em}\raisebox{.4ex}{\relsize{-3}{\textbf{++}}}}
% https://tex.stackexchange.com/questions/28836/typesetting-the-define-equals-symbol
\newcommand{\defeq}{\coloneqq}
\newcommand{\eqdef}{\eqqcolon}
% https://tex.stackexchange.com/questions/5502/how-to-get-a-mid-binary-relation-that-grows
\newcommand{\relmiddle}[1]{\mathrel{}\middle#1\mathrel{}}

% https://tex.stackexchange.com/questions/564216/newcommand-for-each-letter
\ExplSyntaxOn
\NewDocumentCommand{\definealphabet}{mmmm}{
\int_step_inline:nnn{`#3}{`#4}{
\cs_new_protected:cpx{#1 \char_generate:nn{##1}{11}}{
\exp_not:N #2{\char_generate:nn{##1}{11}}}}}
\ExplSyntaxOff

\definealphabet{bb}{\mathbb}{A}{Z}
\definealphabet{rm}{\mathrm}{A}{Z}
\definealphabet{cal}{\mathcal}{A}{Z}
\definealphabet{frak}{\mathfrak}{a}{z}
% \definealphabet{scr}{\mathscr}{A}{Z}
% \definealphabet{frak}{\mathfrak}{A}{Z}

% https://qiita.com/rityo_masu/items/efd44bc8f9229e014237
\allowdisplaybreaks[4]

\usetikzlibrary{
  3d,
  % fit,
  calc,
  math,
  matrix,
  patterns,
  backgrounds,
  arrows.meta,
  shapes.geometric,
  decorations.pathmorphing,
}

% === Beamer Settings ===

% https://uplatexmemo.hatenadiary.jp/entry/2021/01/04/130013
\usepackage{caption}
\captionsetup[table]{justification=centering}
\captionsetup[figure]{justification=centering}

\usetheme{Boadilla}
\usefonttheme{structurebold}
\useinnertheme{circles}
\setbeamerfont{alerted text}{series=\bfseries}
\setbeamerfont{section in toc}{series=\mdseries}
\setbeamerfont{itemize/enumerate body}{size=\large}
\setbeamertemplate{blocks}[rounded]
\setbeamertemplate{navigation symbols}{}
\setbeamertemplate{footline}[frame number]
\setbeamertemplate{title page}{%
    \vspace{2.5em}
    {\usebeamerfont{title} \usebeamercolor[fg]{title} \inserttitle \par}
    {\usebeamerfont{subtitle}\usebeamercolor[fg]{subtitle} \insertsubtitle \par}
    \vspace{1.5em}
    \begin{flushright}
        \usebeamerfont{author}\insertauthor\par
        \usebeamerfont{institute}\insertinstitute \par
        \vspace{3em}
        \usebeamerfont{date}\insertdate\par
        \usebeamercolor[fg]{titlegraphic}\inserttitlegraphic
    \end{flushright}
}

%Beamer Color
\definecolor{blendedblue}{rgb}{0.2,0.2,0.7}
\definecolor{UniBlue}{RGB}{0,150,200} 
\definecolor{UniGreen}{RGB}{0,200,150}
\definecolor{AlertOrange}{RGB}{255,76,0}
\definecolor{AlmostBlack}{RGB}{38,38,38}
\setbeamercolor{normal text}{fg=AlmostBlack}
\setbeamercolor{structure}{fg=UniBlue}
\setbeamercolor{block title}{fg=UniBlue!50!black}
\setbeamercolor{alerted text}{fg=AlertOrange}
\setbeamercolor{itemize item}{fg=black}
\setbeamercolor{itemize subitem}{fg=black}
\setbeamercolor{itemize subsubitem}{fg=black}

% === TITLE ===
\title{\huge{Initial Placement\\for Fruchterman--Reingold force model\\with Coordinate Newton Direction}}
\author{\Large{Hiroki Hamaguchi}}
\institute{\large{5th lab}\\\large{Supervisors: Prof. Akiko Takeda}}
\date{2024/11/29}

\newif\ifShowHidden
% \ShowHiddenfalse
\ShowHiddentrue

\begin{document}

\ifShowHidden
  \maketitle
\fi

\ifShowHidden
  \begin{frame}{Introduction of Graph Drawing}
    \begin{columns}
      \begin{column}{0.58\columnwidth}
        \begin{itemize}
          \item Graph $G = (V,E)$ (vertices $V$  / edges $E$)
          \item \textbf{Graph Visualization} \small{is an important task.}
          \item \textbf{Force-directed graph drawing} \small{is a popular method.}
        \end{itemize}
        \begin{figure}[htbp]
          \centering
          \begin{minipage}{0.55\columnwidth}
            \centering
            \includegraphics[width=\columnwidth]{introExample/social.jpg}
            \caption*{
              Social Network Graph
              \gray{\footnotesize{Designed~by~\href{www.freepik.com}{Freepik}}}
            }
          \end{minipage}%
          \begin{minipage}{0.45\columnwidth}
            \centering
            \includegraphics[width=\columnwidth]{introExample/railway.png}
            \caption*{
              Railroad Graph
              \gray{\footnotesize{By~\href{https://commons.wikimedia.org/wiki/File:High_Speed_Railroad_Map_of_Europe.svg}{Bernese~media}, \href{https://creativecommons.org/licenses/by-sa/3.0}{CC~BY-SA~3.0}}}
            }
          \end{minipage}%
        \end{figure}
      \end{column}
      \begin{column}{0.42\columnwidth}
        \centering
        \begin{figure}[htbp]
          \centering
          \includegraphics[height=0.75\textheight]{introExample/connectedPapers.png}
          \includegraphics[height=0.1\textheight]{introExample/connectedPapersLogo.png}
        \end{figure}
      \end{column}
    \end{columns}
  \end{frame}
\fi

\ifShowHidden
  \begin{frame}{Graph Drawing by NetworkX}
    \begin{columns}
      \begin{column}{0.62\columnwidth}
        \begin{itemize}
          \item NetworkX
                \raisebox{-0.2\baselineskip}{\includegraphics[width=1.5em]{graphDrawing/networkxLogo.png}}
                is a popular Python library.
          \item \texttt{nx.draw} is the default function.
          \item \large{\textbf{Fruchterman--Reingold algorithm}} is~implemented.
        \end{itemize}
        \begin{center}
          \rule{0.8\columnwidth}{0.4pt}
        \end{center}
        \begin{itemize}
          \item<1-> \Large{$\abs{V}=\phantom{0}10$: \phantom{1}0.2 sec / Well Visualized}
          \item<2-> \Large{$\abs{V}=\phantom{0}20$: \phantom{1}0.2 sec / Tangled?}
          \item<3-> \Large{$\abs{V}=500$: \red{11.5} sec / \red{WHAT IS THIS???}}
        \end{itemize}
      \end{column}
      \begin{column}{0.38\columnwidth}
        \begin{figure}[htbp]
          \centering
          \only<1>{\includegraphics[width=\columnwidth]{vscode/10.png}}
          \only<2>{\includegraphics[width=\columnwidth]{vscode/20.png}}
          \only<3>{\includegraphics[width=\columnwidth]{vscode/500.png}}
        \end{figure}
      \end{column}
    \end{columns}
  \end{frame}
\fi

\ifShowHidden
  \begin{frame}{``Twist'' Causes Stagnation}
    \begin{itemize}
      \item
            \textbf{Twist}: Unnecessary edge intersections or tangled structures.
      \item Stagnation of the simulation process.
      \item $\order{\abs{V}^2}$ per iteration. Slow for large-scale graphs.
    \end{itemize}
    \begin{figure}[htbp]
      \centering
      \animategraphics[autoplay,loop,width=0.4\columnwidth]{5}{circle/circle-}{1}{1}
    \end{figure}
  \end{frame}
\fi

\ifShowHidden
  \begin{frame}{Previous Works}
    \begin{columns}
      \begin{column}{0.5\columnwidth}
        \textbf{L-BFGS (Quasi-Newton Method)}
        \begin{itemize}
          \item Numerical optimization approach.
          \item Effective for reducing stress in graph layouts.
        \end{itemize}
        \vspace{0.5cm}
        \textbf{Limitations}
        \begin{itemize}
          \item Treats just as a general optimization problem.
          \item Ignored inherent graph structure.
        \end{itemize}
      \end{column}
      \begin{column}{0.5\columnwidth}
        \textbf{Simulated Annealing (SA)}
        \begin{itemize}
          \item Stochastic optimization method.
          \item Effective for addressing "twist" issues in graphs.
          \item Improves visualization when combined with FR algorithm.
        \end{itemize}
        \vspace{0.5cm}
        \textbf{Limitations}
        \begin{itemize}
          \item Restricted to unweighted graphs.
          \item Limited to circle placement.
          \item Inefficient due to random swapping.
          \item Ignored sparsity of graphs.
        \end{itemize}
      \end{column}
    \end{columns}
  \end{frame}
\fi

\ifShowHidden
  \begin{frame}{Our Contribution}
    \begin{figure}[h]
      \centering
      \includegraphics[width=0.75\columnwidth]{../main/fig1/fig1_slide.pdf}
      \caption{\texttt{jagmesh1} dataset after 50 iterations.}
    \end{figure}
  \end{frame}
\fi

\begin{frame}{Definition of FR Layout}
  \begin{columns}
    \begin{column}{0.7\columnwidth}
      Historically, FR layout seeks an \textbf{equilibrium of two forces}:
      \begin{equation*}
        \tccA{F_{i,j}^\mathrm{a}(d) \defeq \frac{w_{i,j} d^2}{k}}, \quad \tccD{F^\mathrm{r}(d) \defeq -\frac{k^2}{d}}.
      \end{equation*}
      Its scalar potential, energy, is defined as
      \begin{gather*}
        \tccA{E_{i,j}^\mathrm{a}(d) \defeq \int_{0}^{d} F_{i,j}^\mathrm{a}(r) \dd{r} = \frac{w_{i,j} d^3}{3k}}, \quad
        \tccD{E^\mathrm{r}(d)       \defeq \int_{\infty}^{d} F^\mathrm{r}(r) \dd{r} = -k^2\log{d}}, \\
        E_{i,j}(d)            \defeq \tccA{E_{i,j}^\mathrm{a}(d)} + \tccD{E^\mathrm{r}(d)}.
      \end{gather*}
      Seek equilibrium $\leftrightarrow$ \textbf{find local minimum of $f(X)$} (\textbf{non}-convex):
      \begin{mini}
        {X \in \bbR^{2 \times n}}
        {f(X) \defeq \sum_{i<j} E_{i,j}(\norm{x_i - x_j}).}
        {\label{eq:fr}}
        {}
      \end{mini}
    \end{column}
    \begin{column}{0.3\columnwidth}
      \begin{figure}[h]
        \centering
        \includegraphics[width=\columnwidth]{../main/fr_layout/fr_layout.pdf}
      \end{figure}
      \begin{figure}[h]
        \centering
        \includegraphics[width=\columnwidth]{../main/energy_3d/energy_3d.png}
      \end{figure}
    \end{column}
  \end{columns}
\end{frame}

\begin{frame}{Newton's Method}
  \begin{itemize}
    \item $f\colon \bbR^n \to \bbR$ : strictly convex function /  The second order approximation at $x_0$ is
          \begin{equation*}
            f(x) \approx f(x_0) + \nabla f(x_0)^\top (x - x_0) + \frac{1}{2} (x - x_0)^\top \nabla^2 f(x_0) (x - x_0).
          \end{equation*}
    \item  argmin $x^*$ satisfies
          \begin{align*}
                 & \nabla f(x_0) + \nabla^2 f(x_0) (x^* - x_0) = 0 \\
            \iff & x^* = x_0 - \nabla^2 f(x_0)^{-1} \nabla f(x_0).
          \end{align*}
    \item  $d = -\nabla^2 f(x_0)^{-1} \nabla f(x_0)$: \textbf{the Newton direction}
          (\textbf{too expensive} ... )
    \item coordinate Newton direction: $d_i = -\nabla^2 f_i(x_i)^{-1} \nabla f_i(x_i)$
    \item Computable!
  \end{itemize}
\end{frame}

\begin{frame}{Reduction to Discrete}
  Graphs have sparsity : $\tccA{\abs{E}} \ll \tccD{\abs{V}^2}$ / the energy function $E_{i,j}$ is $\tccD{-k^2\log{d}}$ for all $(i,j) \notin E$.
  \begin{mini}
    {X \in \bbR^{2 \times n}}
    {\sum_{(i,j)\in E} \tccA{\frac{w_{i,j}\norm{x_i - x_j}^3}{3k}} - \tccD{\sum_{i<j} k^2\log{\norm{x_i - x_j}}}.}
    {\label{eq:frApprox}}
    {}
  \end{mini}
  Converting the second term into a constraint
  (obj: $\tccA{\order{\abs{E}}}$ terms from $\tccD{\order{\abs{V}^2}}$ terms)
  \begin{mini}
    {X \in \bbR^{2 \times n}}
    {f^{\mathrm{a}}(X) \defeq \tccA{\sum_{(i,j)\in E} \frac{w_{i,j}\norm{x_i - x_j}^3}{3k}}}
    {\label{eq:frApprox2}}
    {}
    \addConstraint{\tccD{\norm{x_i - x_j}}}{\tccD{\geq \epsilon},\quad}{\forall (i,j)\,(i<j)}
  \end{mini}
  where $\epsilon$ is a suitably chosen positive constant.\\
  This conversion does not lose the essence of the problem too much because $\tccD{E^\mathrm{r}(d)=-k^2\log{d}}$ is monotonically decreasing.
\end{frame}

\begin{frame}{Discrete Optimization}
  \begin{columns}
    \begin{column}{0.47\columnwidth}
      problem \eqref{eq:frApprox2} still involves $\order{\abs{V}^2}$ constraints.
      By \textbf{fixing} the possible point placement to $Q$ (any two points apart $\epsilon$), we can \textbf{skip the check of constraints} in the problem~\eqref{eq:frApprox2}.
      \begin{mini}
        {\pi\colon V \to \tccD{Q}}
        {\tccA{\sum_{(i,j)\in E} \frac{w_{i,j}\norm{\pi(v_i) - \pi(v_j)}^3}{3k}},}
        {\label{eq:frApprox3}}
        {}
        \addConstraint{\tccD{\pi \text{ is injective}}.}
      \end{mini}
      Skipping the check of $\order{\abs{V}^2}$ constraints\\
      $\to$ Reducing the complexity to $\order{\abs{E}}$\\
      $\to$ \textbf{speedup}.
    \end{column}
    \begin{column}{0.53\columnwidth}
      \begin{figure}[t]
        \centering
        \includegraphics[width=\columnwidth]{../main/pi/pi.pdf}
      \end{figure}
    \end{column}
  \end{columns}
\end{frame}

\begin{frame}{Proposed Algorithm 1}
  We solve the problem~\eqref{eq:frApprox3} using the coordinate Newton direction.

  Let the objective function $f^{\mathrm{a}}_i(x_i)$ corresponding to a vertex $v_i$ be
  \begin{equation*}
    f^{\mathrm{a}}_i(x_i) \defeq \sum_{j \neq i} \frac{w_{i,j}\norm{x_i - x_j}^3}{3k}.
  \end{equation*}
  Its gradient and Hessian matrix are
  \begin{align*}
    \nabla f^{\mathrm{a}}_i(x_i)   & = \sum_{j \neq i} \frac{w_{i,j}\norm{x_i - x_j}}{k} (x_i - x_j),     \\
    \nabla^2 f^{\mathrm{a}}_i(x_i) & = \sum_{j \neq i} \frac{w_{i,j}\norm{x_i - x_j}}{k} \mqty(1      & 0 \\0&1) + \sum_{j \neq i} \frac{w_{i,j}}{k\norm{x_i - x_j}} (x_i - x_j)(x_i - x_j)^\top.
  \end{align*}
  Since $f^{\mathrm{a}}_i$ is convex, we can use the coordinate Newton direction.
  This is a large difference from the functions $f_i(x_i)$ in Eq.~\eqref{eq:fi} and $f^{\mathrm{a}}(X)$ in Prob.~\eqref{eq:frApprox2}, which are not convex.
\end{frame}

\begin{frame}{proposed algorithm 2}
  The ordinary updated rule with the coordinate Newton direction is
  \begin{equation*}
    x_i - \nabla^2 f^{\mathrm{a}}_i(x_i)^{-1} \nabla f^{\mathrm{a}}_i(x_i).
  \end{equation*}
  $x_i^\mathrm{new}$ may not be in the hexagonal lattice $Q^\mathrm{hex}$.
  Thus, we must round to the nearest point in $Q^\mathrm{hex}$.
  We empirically found that adding a random vector to the Newton direction is effective.
  % for the optimization process, which is a similar strategy to the SA in Sec.~\ref{ssec:preprocessing}. This randomness can help to escape from local minima and to explore the solution space more effectively.
  % In conclusion, the updated rule for the vertex $v_i$ is
  \begin{equation*}
    x_i^\mathrm{new} \gets \mathrm{round}\qty(x_i - \nabla^2 f^{\mathrm{a}}_i(x_i)^{-1} \nabla f^{\mathrm{a}}_i(x_i) + t \cdot \text{rand}),
  \end{equation*}
  where $\mathrm{round}(\hat{x})$ denotes the operation assigning $\hat{x}$ to the nearest point in the hexagonal lattice $Q^\mathrm{hex}$, $\mathrm{rand}$ is a random vector with a unit norm, and $t$ is a parameter controlling the randomness.

  \begin{figure}[t]
    \centering
    \includegraphics[width=0.5\columnwidth]{../main/hex/hex.pdf}
    \caption{Visual explanation of the one iteration of the proposed algorithm. Step1. Compute the coordinate Newton direction for a randomly selected vertex (blue). Step2. Decide $x_i^\mathrm{new}$ by rounding the direction and adding a random vector. Step3. Move the vertex and swap the vertices if there is a collision. In this case, swap blue and green vertices.}
    \label{fig:hex}
  \end{figure}
\end{frame}

\begin{frame}{optimal Scaling}
  We can find optimal scaling factor $c^*$.
  We scale $X = (x_1, \dots, x_n)$ as $x_i \gets c x_i$ for all $i$.\\
  This problem is to minimize $\phi(c)$:
  \begin{equation*}
    \phi(c) \defeq \qty(\sum_{(i,j) \in E} \frac{w_{i,j} (c \norm{x_i - x_j})^3}{3k}) - k^2 \sum_{i < j} \log(c \norm{x_i - x_j})
  \end{equation*}
  The function $\phi(c)$ is convex, and we can find the optimal scaling factor $c^*$ by
  \begin{equation}\label{eq:scaling}
    c^* = \qty(\frac{k^2 n(n-1)}{3 \sum_{(i,j) \in E} \frac{w_{i,j} \norm{x_i - x_j}^3}{k}})^{1/3}.
  \end{equation}
  This value can be computed in $\order{\abs{E}}$ complexity.

  Notably, the optimal solution to Prob.~\eqref{eq:frApprox3} is invariant under scaling. Thus, we can select any $\epsilon$ to define the hexagonal lattice $Q^\mathrm{hex}$ as far as we scale the placement by $c^*$ as post-processing.
\end{frame}

\begin{frame}[fragile]{Pseudo Code}
  \begin{center}
    \begin{minipage}{0.7\columnwidth}
      \vspace{-0.2cm}
      \begin{algorithm}[H]
        \caption{Proposed algorithm as initial placement}
        \label{alg:proposed}
        % \KwIn{Graph $G_W = (V, E)$}
        % \KwOut{Initial placement $X = (x_1, \dots, x_n)$}
        Define parameters $N_\mathrm{iter}^\mathsf{CN}, t, \Delta t$ and hexagonal lattice $Q^\mathrm{hex}$\;
        Set $\pi$ as a injection from $V$ to $Q^\mathrm{hex}$ randomly\;
        \For{$j \gets 0$ \KwTo $N_\mathrm{iter}^\mathsf{CN}$}{
        $v_i \gets \text{randomly selected vertex from } V$\;
        $x_i \gets \pi(v_i)$\;
        $x_i^\mathrm{new} \gets \mathrm{round}(x_i - \nabla^2 f_i(x_i)^{-1} \nabla f_i(x_i) + t \cdot \mathrm{rand})$\;
          \If{$\exists v_j \st \pi(v_j) = x_i^\mathrm{new}$}{
            Swap $v_i$ and $v_j$ in $\pi$\;
          } \Else{
            $\pi(v_i) \gets x_i^\mathrm{new}$\;
          }
          }
        $x_i \gets \pi(v_i)$ for all $v_i \in V$\;
        $c^* \gets \text{optimal scaling factor by Eq.~\eqref{eq:scaling}}$\;
        $x_i \gets c^* x_i$ for all $v_i \in V$\;
          \Return $X$
      \end{algorithm}
    \end{minipage}
  \end{center}
\end{frame}

\begin{frame}{Results 1}
  \begin{figure}[h]
    \centering
    \addtolength{\tabcolsep}{-0.5em}
    \begin{tabular}{cccccc}
      \multicolumn{6}{c}{\textbf{\texttt{jagmesh1}} $(\abs{V}=936, \abs{E}=2664, \text{sparsity}=0.609\text{\%})$ \quad Figures are at 50 iterations.}    \\
      \raisebox{-.5\height}{\includegraphics[width=0.275\columnwidth]{../main/individual/plot/jagmesh1.pdf}} &
      \makecell{\small{\textsf{FR}}                                                                                                                       \\[-0.2em]\includegraphics[width=0.135\columnwidth]{../main/individual/vis/jagmesh1_FR.png}} &
      \makecell{\small{\textsf{L-BFGS}}                                                                                                                   \\[-0.2em]\includegraphics[width=0.135\columnwidth]{../main/individual/vis/jagmesh1_L-BFGS.png}} &
      \makecell{\small{\textsf{\textbf{CN}-FR}}                                                                                                           \\[-0.2em]\includegraphics[width=0.135\columnwidth]{../main/individual/vis/jagmesh1_CN-FR.png}} &
      \makecell{\small{\textsf{\textbf{CN}-L-BFGS}}                                                                                                       \\[-0.2em]\includegraphics[width=0.135\columnwidth]{../main/individual/vis/jagmesh1_CN-L-BFGS.png}} &
      \makecell{\small{\textsf{BEST}}                                                                                                                     \\[-0.2em]\includegraphics[width=0.135\columnwidth]{../main/individual/vis/opt_jagmesh1.png}} \\

      \multicolumn{6}{c}{\textbf{\texttt{dwt\_1005}} $(\abs{V}=1005, \abs{E}=3808, \text{sparsity}=0.755\text{\%})$ \quad Figures are at 100 iterations.} \\
      \raisebox{-.5\height}{\includegraphics[width=0.275\columnwidth]{../main/individual/plot/dwt_1005.pdf}} &
      \makecell{\small{\textsf{FR}}                                                                                                                       \\[-0.2em]\includegraphics[width=0.135\columnwidth]{../main/individual/vis/dwt_1005_FR.png}} &
      \makecell{\small{\textsf{L-BFGS}}                                                                                                                   \\[-0.2em]\includegraphics[width=0.135\columnwidth]{../main/individual/vis/dwt_1005_L-BFGS.png}} &
      \makecell{\small{\textsf{\textbf{CN}-FR}}                                                                                                           \\[-0.2em]\includegraphics[width=0.135\columnwidth]{../main/individual/vis/dwt_1005_CN-FR.png}} &
      \makecell{\small{\textsf{\textbf{CN}-L-BFGS}}                                                                                                       \\[-0.2em]\includegraphics[width=0.135\columnwidth]{../main/individual/vis/dwt_1005_CN-L-BFGS.png}} &
      \makecell{\small{\textsf{BEST}}                                                                                                                     \\[-0.2em]\includegraphics[width=0.135\columnwidth]{../main/individual/vis/opt_dwt_1005.png}} \\
    \end{tabular}
  \end{figure}
\end{frame}

\begin{frame}{Results 2}
  \begin{figure}[h]
    \centering
    \addtolength{\tabcolsep}{-0.5em}
    \begin{tabular}{cccccc}
      \multicolumn{6}{c}{\textbf{\texttt{dwt\_2680}} $(\abs{V}=2680, \abs{E}=11173, \text{sparsity}=0.311\text{\%})$ \quad Figures are at 150 iterations.} \\
      \raisebox{-.5\height}{\includegraphics[width=0.275\columnwidth]{../main/individual/plot/dwt_2680.pdf}} &
      \makecell{\small{\textsf{FR}}                                                                                                                        \\[-0.2em]\includegraphics[width=0.135\columnwidth]{../main/individual/vis/dwt_2680_FR.png}} &
      \makecell{\small{\textsf{L-BFGS}}                                                                                                                    \\[-0.2em]\includegraphics[width=0.135\columnwidth]{../main/individual/vis/dwt_2680_L-BFGS.png}} &
      \makecell{\small{\textsf{\textbf{CN}-FR}}                                                                                                            \\[-0.2em]\includegraphics[width=0.135\columnwidth]{../main/individual/vis/dwt_2680_CN-FR.png}} &
      \makecell{\small{\textsf{\textbf{CN}-L-BFGS}}                                                                                                        \\[-0.2em]\includegraphics[width=0.135\columnwidth]{../main/individual/vis/dwt_2680_CN-L-BFGS.png}} &
      \makecell{\small{\textsf{BEST}}                                                                                                                      \\[-0.2em]\includegraphics[width=0.135\columnwidth]{../main/individual/vis/opt_dwt_2680.png}} \\

      \multicolumn{6}{c}{\textbf{\texttt{3elt}} $(\abs{V}=4720, \abs{E}=13722, \text{sparsity}=0.123\text{\%})$ \quad Figures are at 150 iterations.}      \\
      \raisebox{-.5\height}{\includegraphics[width=0.275\columnwidth]{../main/individual/plot/3elt.pdf}}     &
      \makecell{\small{\textsf{FR}}                                                                                                                        \\[-0.2em]\includegraphics[width=0.135\columnwidth]{../main/individual/vis/3elt_FR.png}} &
      \makecell{\small{\textsf{L-BFGS}}                                                                                                                    \\[-0.2em]\includegraphics[width=0.135\columnwidth]{../main/individual/vis/3elt_L-BFGS.png}} &
      \makecell{\small{\textsf{\textbf{CN}-FR}}                                                                                                            \\[-0.2em]\includegraphics[width=0.135\columnwidth]{../main/individual/vis/3elt_CN-FR.png}} &
      \makecell{\small{\textsf{\textbf{CN}-L-BFGS}}                                                                                                        \\[-0.2em]\includegraphics[width=0.135\columnwidth]{../main/individual/vis/3elt_CN-L-BFGS.png}} &
      \makecell{\small{\textsf{BEST}}                                                                                                                      \\[-0.2em]\includegraphics[width=0.135\columnwidth]{../main/individual/vis/opt_3elt.png}} \\
    \end{tabular}
  \end{figure}
\end{frame}

% \ifShowHidden
%   \begin{frame}{Algorithm for FR Layout}
%     \begin{columns}
%       \begin{column}{0.4\columnwidth}
%         \textbf{``spring\_layout'' in NetworkX\\and ``fdp'' in Graphviz}
%         \begin{itemize}
%           \item Adaptive cooling scheme
%           \item \textbf{gradient descent method}\\with constant step size\\per each vertex
%           \item strong \textbf{theoretical background}
%           \item \cite{tunkelang1999numerical}
%         \end{itemize}
%       \end{column}
%       \begin{column}{0.6\columnwidth}
%         \begin{figure}[htbp]
%           \centering
%           \includegraphics[width=\columnwidth]{imgs/FR_code.png}
%           \caption{\cite{kobourov2012spring}}
%         \end{figure}
%       \end{column}
%     \end{columns}
%   \end{frame}
% \fi

% \ifShowHidden
%   \begin{frame}{Key Technique 1: Separation of the objective function}
%     \begin{align*}
%       f(X) =                      & \sum_{i<j} f_{i,j}(d_{i,j}) = \cAText{\overset{\text{attraction}}{\underbrace{\sum_{(i,j)\in E} f^a_{i,j}(d_{i,j})}_{\text{sparse}}}} + \cDText{\overset{\text{repulsion}}{\underbrace{\sum_{i<j} f^r_{i,j}(d_{i,j})}_{\text{dense}}}} \\
%       \to \mathrm{minimize} \quad & \cAText{\sum_{(i,j)\in E} f^a_{i,j}(d_{i,j})} \quad \mathrm{subject \enspace to} \quad \cDText{f^r_{i,j}(d_{i,j}) \leq \epsilon, \quad i<j}                                                                                            \\
%       \to \mathrm{minimize} \quad & \cAText{\sum_{(i,j)\in E} f^a_{i,j}(d_{i,j})} \quad \mathrm{subject \enspace to} \quad \cDText{\text{each $x_i$ has an exclusive $\epsilon'$-ball}}
%     \end{align*}
%     \begin{itemize}
%       \item Without constraints, $X=0$ is the optimal solution $\to$ \textbf{closest packing}
%       \item With \textbf{hexagonal close-packed} structure,
%             $f$ is the sum of $\abs{V}^2 \to \abs{E}$ terms
%       \item partially overlapped with \cite{ghassemitoosiSimulatedAnnealingPreProcessing2016,s22145179}
%     \end{itemize}
%   \end{frame}
% \fi

% \ifShowHidden
%   \begin{frame}{Key Technique 2: Randomized subspace algorithm}
%     \begin{itemize}
%       \item Researches on Randomized Subspace Newton (\textbf{RSN}) are conducted.
%             \begin{itemize}
%               \item Proposed in~\cite{NEURIPS2019_bc6dc48b}
%               \item Extended in~\cite{cartisRandomisedSubspaceMethods2022},\cite{fujiRandomizedSubspaceRegularized2022},\cite{higuchiFastConvergenceSecondOrder2024}
%               \item Analysis is based on~\cite{karimireddyGlobalLinearConvergence2018}
%             \end{itemize}
%     \end{itemize}
%     \begin{columns}[T]
%       \column{0.5\columnwidth}
%       \begin{itemize}
%         \item Apply some second-order method to subspaces
%         \item graph drawing: optimize over $\prod_{i=1}^{n} \mathbb{R}^2$\\
%               $\to$ suitable for \textbf{RSN}!
%       \end{itemize}
%       \column{0.5\columnwidth}
%       \begin{figure}[htbp]
%         \centering
%         \includegraphics[width=0.9\columnwidth]{imgs/randomSubspace.pdf}
%       \end{figure}
%     \end{columns}
%   \end{frame}
% \fi

% % \ifShowHidden
% \begin{frame}{Advantages of Randomized Subspace Newton}
%   \begin{figure}[htbp]
%     \centering
%     \begin{tikzpicture}[scale=1.5]
%       \coordinate (A) at (0,0);
%       \coordinate (B) at (3,0);
%       \coordinate (C) at (0,2);

%       \draw[decorate, decoration={coil, segment length=4, aspect=0.5}] (A) -- (B);
%       \draw[decorate, decoration={coil, segment length=4, aspect=0.5}] (A) -- (C);

%       \draw[gray, dashed, ultra thick] (A) -- ++(9/2.8,4/2.8);
%       \draw[red, ultra thick, -] (A) -- ++(1.125,0.57142);
%       \node[star, star points=5, star point ratio=2.25, fill=red,scale=0.5] at ($(A)+(1.125,0.57142)$) {};
%       \node[anchor=west,align=left,gray] at ($(A)+(9/2.8+0.3,4/2.8)$) {\Large{$g=\nabla f$}\\[0.4em]\Large{Descent direction}};
%       \node[anchor=west,align=left,red] at (9/2.8+0.3,0.5) {\Large{$d=(\nabla^2 f)^{-1}\nabla f$}\\[0.4em]\Large{Newton's direction}};

%       \coordinate (opt) at ($(B)!0.5!(C)$);
%       \node[star, star points=5, star point ratio=2.25, fill=cA, minimum size=5pt] at (opt) {};
%       \node[above=0.5cm] at (opt) {\large{\cAText{optimal}}};

%       \filldraw[red] (A) circle (2pt);
%       \filldraw[black] (B) circle (2pt);
%       \filldraw[black] (C) circle (2pt);
%     \end{tikzpicture}
%   \end{figure}
%   \begin{itemize}
%     \item random subspace problem (subproblem for vertex $i$):
%           \begin{equation*}
%             \min_{x_i \in \mathbb{R}^2} \sum_{j \in \mathrm{Adj}(i)} \frac{a_{i,j} \norm{x_i - x_j}_2^3}{3k} \quad \leftarrow \text{\textbf{CONVEX} optimization}
%           \end{equation*}
%     \item Original problem, subspace problem, and restricted problem are all \textbf{NON-CONVEX}
%     \item All distance $d_{i,j}=\norm{x_i-x_j}_2 \geq \epsilon'$ $\to$ numerically stable
%   \end{itemize}
% \end{frame}
% % \fi

% \ifShowHidden
%   \begin{frame}{proposed algorithm 2}
%     \begin{algorithm}[H]
%       \SetAlgoLined
%       \KwIn{Set of indices $S$, initial vectors $v_i$}
%       \KwOut{Updated vector $v_i$}
%       Initialize $X$ randomly\;
%       \While{not converged}{
%         Randomly select an index $i \in S$ except for the taboo list\;
%         Compute Newton's direction $d_i=(\nabla^2 f_i)^{-1}\nabla f_i$\;
%         Update $v_i$ subject to the constraints\;
%         \If{not updated}{
%           Add $i$ to the taboo list\;
%         }
%       }
%       \Return $X$\;
%       \caption{Randomized Newton Direction Update for Vector $v_i$}
%     \end{algorithm}
%   \end{frame}
% \fi

\ifShowHidden
  \begin{frame}{Applicatoin1}
    \begin{columns}
      \begin{column}{0.4\columnwidth}
        \textbf{``sfdp'' in Graphviz}
        \begin{itemize}
          \item Scalable Force-Directed Placement
          \item \textbf{Multilevel} approach
          \item Barnes--Hut algorithm (Q-tree)\cite{Hu2006EfficientHF}
                \cite{barnesHierarchicalLogForcecalculation1986}
          \item These methods are\\\textbf{out of scope},\\
                but our work is still \textbf{applicable}.
        \end{itemize}
      \end{column}
      \begin{column}{0.6\columnwidth}
        \begin{figure}[htbp]
          \centering
          \includegraphics[width=0.7\columnwidth]{imgs/BH.png}
          \caption{\footnotesize{\url{https://jheer.github.io/barnes-hut/}}}
        \end{figure}
      \end{column}
    \end{columns}

    % This paper has demonstrated the effectiveness of the proposed method on various graphs, but we can also apply it to larger-scale problems.
    % For instance, the Scalable Force-Directed Placement (sfdp) of Graphviz~\cite{ellsonGraphvizOpenSource2002}, based on \cite{Hu2006EfficientHF}, employs a multilevel approach to accelerate processing for larger graphs by progressively coarsening vertices.

    % The coarsening operation in sfdp does not critically conflict with the proposed method, making it feasible to combine both approaches.
    % Specifically, by iteratively applying the proposed method to the entire coarsened graph or groups of vertices consolidated through coarsening, it is possible to extend its applicability to larger-scale problems.
    % We can expect this approach to yield faster and higher-quality solutions.
    % Addressing this integration is one of the challenges for future research.
  \end{frame}
\fi

\ifShowHidden
  \begin{frame}{Application2}
    \begin{columns}
      \begin{column}{0.5\columnwidth}
        \textbf{``objective functions arising from graphs''}~\cite{recht-wright}
        \begin{itemize}
          \item \textbf{stochastic coordinate descent}~\cite{recht-wright} is a popular method.
          \item Can we use coordinate Newton direction?
        \end{itemize}
        \begin{mini*}
          {X \in \bbR^{2 \times n}}
          {f(X) = \sum_{(i,j)\in E} f_{i,j}(x_i, x_j)+\lambda \sum_{i=1}^{n} \Omega_i(x_i),}
          {}
          {}
        \end{mini*}
      \end{column}
      \begin{column}{0.5\columnwidth}
        \textbf{graph isomorphism problem}
        \begin{itemize}
          \item When we relax it to a continuous optimization problem on Riemannian manifolds~\cite{klusContinuousOptimizationMethods2023,klusContinuousOptimizationMethods2023}
                % \addConstraint{Q}{\in \left\{ Q \in \mathbb{R}^{n \times n} \relmiddle| Q^\top Q = I_n \right\}}
          \item  Drawing graph symmetry is at least as difficult as the graph isomorphism problem~\cite{eades1984heuristic}.
          \item Can we utilize coordinate Newton direction to Riemanian optimization?
        \end{itemize}
        \begin{figure}[t]
          \centering
          \includegraphics[width=\columnwidth]{../main/iso/iso.pdf}
        \end{figure}
      \end{column}
    \end{columns}
  \end{frame}
\fi

\ifShowHidden
  \begin{frame}{End of the Presentation}
    \begin{block}{Summary}
      \begin{itemize}
        \item Fruchterman--Reingold layout is a tough problem
        \item Hexagonal Lattice + coordinate Newton direction
        \item initialization + L-BFGS gives the best result
      \end{itemize}
    \end{block}
    \begin{block}{Acknowledgement}
      \large{
        I would like to express my sincere gratitude to \textbf{Pierre-Louis Poirion}, \textbf{Andi Han}, and \textbf{Naoki Marumo}.
      }
    \end{block}
  \end{frame}
\fi

\setbeamercolor{structure}{fg=blendedblue}
\setbeamercolor{block title}{fg=blendedblue!50!black}

\ifShowHidden
  \begin{frame}[allowframebreaks]{Reference}
    \scriptsize
    \beamertemplatetextbibitems
    \bibliographystyle{hapalike}
    \bibliography{../FruchtermanReingoldByRandomSubspace.bib}
    % \begin{thebibliography}{1}
    % \end{thebibliography}
  \end{frame}
\fi

\end{document}