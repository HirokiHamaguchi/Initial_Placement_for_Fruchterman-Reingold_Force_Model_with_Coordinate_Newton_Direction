\documentclass[journal]{IEEEtran}

\usepackage{algorithm}      % \begin{algorithm}
\usepackage{algorithmic}    % \begin{algorithmic}
\usepackage{amsmath}        % \begin{align*}
\usepackage{amssymb}        % \mathbb{A}
\usepackage{amsthm}         % \newtheorem
\usepackage{ascmac}         % \begin{screen}
\usepackage{bm,bbm}         % \bm{A}, \bbm{1}
\usepackage{booktabs}       % \toprule, \midrule, \bottomrule
\usepackage{caption}        % \captionsetup
\usepackage{enumitem}       % \begin{enumerate}[label=(\alph*)]
\usepackage{geometry}       % \geometry{margin=1in}
\usepackage{hyperref}       % \href{URL}{text}
\usepackage{ifthen}         % \ifthenelse
\usepackage{mathrsfs}       % \mathscr{A}
\usepackage{mathtools}      % \mathrlap
\usepackage{optidef}        % \begin{mini*}{x}{f(x)}{}{}
\usepackage{orcidlink}      % \orcidlink
\usepackage{physics}        % \qty, \norm, \abs
\usepackage{subfiles}       % \subfile{file}
\usepackage{thm-restate}    % \begin{restatable}{theorem}{thm}
\usepackage{tikz}           % \begin{tikzpicture}
\usepackage{xparse}         % \NewDocumentCommand
% \usepackage{calc}         % \setlength
% \usepackage{cancel}       % \cancel
% \usepackage{parskip}      % \setlength{\parskip}{0.5em}
% \usepackage{csvsimple}    % \csvautotabular
% \usepackage{diagbox}      % \diagbox
% \usepackage{dsfont}       % \mathds{1}
% \usepackage{epsfig}       % \epsfig
% \usepackage{fancybx}      % \ovalbox
% \usepackage{float}        % \begin{figure}[H]
% \usepackage{lipsum}       % \lipsum
% \usepackage{listings}     % \begin{lstlisting}
% \usepackage{makecell}     % \makecell{L1\L2}
% \usepackage{multicol}     % \begin{multicols}{2}
% \usepackage{multirow}     % \multirow
% \usepackage{nicematrix}   % \begin{NiceMatrix}
% \usepackage{qcircuit}     % \Qcircuit
% \usepackage{siunitx}      % \SI{1}{\second}
% \usepackage{stfloats}     % \begin{figure*}
% \usepackage{subcaption}   % \begin{subfigure}
% \usepackage{ulem}         % \sout
% \usepackage[hyphens]{url} % \url
% \usepackage{wrapfig}      % \begin{wrapfigure}
% \usepackage[all]{xy}      % \xymatrix
% \usepackage[dvipdfmx]{graphicx}
% \usepackage[square, sort, comma, numbers]{natbib}

\geometry{margin=1in}
\hypersetup{colorlinks=true,linkcolor=blue,citecolor=blue,urlcolor=blue}

\definecolor{cA}{HTML}{0072BD}
\definecolor{cB}{HTML}{EDB120}
\definecolor{cC}{HTML}{77AC30}
\definecolor{cD}{HTML}{D95319}

\newcommand{\red}[1]{\textcolor{red}{#1}}
\newcommand{\blue}[1]{\textcolor{blue}{#1}}
\newcommand{\cyan}[1]{\textcolor{cyan}{#1}}
\newcommand{\gray}[1]{\textcolor{gray}{#1}}
\newcommand{\green}[1]{\textcolor{green}{#1}}
\newcommand{\brown}[1]{\textcolor{brown}{#1}}
\newcommand{\black}[1]{\textcolor{black}{#1}}
\newcommand{\st}{\text{ s.t. }}
\newcommand{\Img}[1]{\mathrm{Im}\qty(#1)}
\newcommand{\Ker}[1]{\mathrm{Ker}\qty(#1)}
\newcommand{\Supp}[1]{\mathrm{supp}\qty(#1)}
\newcommand{\Rank}[1]{\mathrm{rank}\qty(#1)}
\newcommand{\floor}[1]{\left\lfloor #1 \right\rfloor}
\newcommand{\ceil}[1]{\left\lceil #1 \right\rceil}
% C++ (https://tex.stackexchange.com/questions/4302/prettiest-way-to-typeset-c-cplusplus)
\newcommand{\Cpp}{C\nolinebreak[4]\hspace{-.05em}\raisebox{.4ex}{\relsize{-3}{\textbf{++}}}}
% https://tex.stackexchange.com/questions/28836/typesetting-the-define-equals-symbol
\newcommand{\defeq}{\coloneqq}
\newcommand{\eqdef}{\eqqcolon}
% https://tex.stackexchange.com/questions/5502/how-to-get-a-mid-binary-relation-that-grows
\newcommand{\relmiddle}[1]{\mathrel{}\middle#1\mathrel{}}

\DeclareMathOperator{\Proj}{Proj}
\DeclareMathOperator{\Exp}{Exp}
\DeclareMathOperator{\Hess}{Hess}
\DeclareMathOperator{\Retr}{Retr}
\DeclareMathOperator{\Span}{span}
% \DeclareMathOperator{\myGrad}{grad}
% \renewcommand{\grad}{\myGrad}

% https://tex.stackexchange.com/questions/564216/newcommand-for-each-letter
\ExplSyntaxOn
\NewDocumentCommand{\definealphabet}{mmmm}{
\int_step_inline:nnn{`#3}{`#4}{
\cs_new_protected:cpx{#1 \char_generate:nn{##1}{11}}{
\exp_not:N #2{\char_generate:nn{##1}{11}}}}}
\ExplSyntaxOff

\definealphabet{bb}{\mathbb}{A}{Z}
\definealphabet{rm}{\mathrm}{A}{Z}
\definealphabet{cal}{\mathcal}{A}{Z}
\definealphabet{frak}{\mathfrak}{a}{z}
% \definealphabet{scr}{\mathscr}{A}{Z}
% \definealphabet{frak}{\mathfrak}{A}{Z}

\newtheorem{theorem}{Theorem}
\newtheorem{proposition}{Proposition}
\newtheorem{lemma}{Lemma}
\newtheorem{definition}{Definition}
\newtheorem{corollary}{Corollary}
\newtheorem{remark}{Remark}
\newtheorem{example}{Example}

% https://qiita.com/rityo_masu/items/efd44bc8f9229e014237
\allowdisplaybreaks[4]

% \lstset{
%   language=Python,numbers=left,frame=single,breaklines=true,lineskip=-0.9ex,xleftmargin=3zw,xrightmargin=0zw,
%   basicstyle=\ttfamily,ndkeywordstyle=\small,identifierstyle=\small,numberstyle=\scriptsize,
%   commentstyle=\color[rgb]{0,0.6,0},stringstyle=\small\ttfamily\color[rgb]{0.89,0.55,0},keywordstyle=\small\bfseries\color[rgb]{0.28,0.28,0.95},
% }

\usetikzlibrary{
  3d,
  calc,
  math,
  matrix,
  patterns,
  backgrounds,
  arrows.meta,
}

% \graphicspath{{./fig/}}

% \providecommand{\main}{.}
\newboolean{isMain}
\setboolean{isMain}{true}

\begin{document}

\title{Faster Fruchterman--Reingold Algorithm by Random Subspace Method}
\author{Hiroki Hamaguchi\,\orcidlink{0009-0005-7348-1356}}
\date{\today}
\maketitle

\begin{abstract}
  The abstract goes here.
\end{abstract}

\begin{IEEEkeywords}
  Graph drawing, Fruchterman--Reingold algorithm, Random Subspace method.
\end{IEEEkeywords}

\section{Introduction}

\IEEEPARstart{D}{rawing} graph is one of the most fundamental task in computer science.

One of the graph drawing algorithms is
force-directed algorithm.
Davidson and Harel proposed algorithm using simulated annealing~\cite{davidson1996drawing}.
Kamada and Kawai~\cite{kamadaAlgorithmDrawingGeneral1989} proposed an algorithm.
Fruchterman--Reingold algorithm~\cite{fruchtermanGraphDrawingForcedirected1991}, as known as spring embedder algorithm, is one of the most popular algorithm.
The algorithm is based on the physical model of a system of particles and springs.

To mitigate this heavy computational cost, several methods have been proposed.

One of the strategies is to approximate the $N$-body simulation by a hierarchical method, such as the fast multipole method~\cite{greengardFastAlgorithmParticle1987}, the Barnes--Hut approximation~\cite{barnesHierarchicalLogForcecalculation1986}, and multilevel approach~\cite{Hu2006EfficientHF} or stress majorization~\cite{gansnerGraphDrawingStress2005}.

The other method is to just speed up the optimization algorithm as it is, which has the same spirit of our work.
use Stochastic Gradient Decent~\cite{8419285}.
GPU parallel architecture
\cite{gajdosParallelFruchtermanReingold2016},
and numerical optimization techniques~\cite{6183577}.

In this paper, we will use the Random Subspace method~\cite{NEURIPS2019_bc6dc48b,fujiRandomizedSubspaceRegularized2022,cartisRandomisedSubspaceMethods2022,higuchiFastConvergenceSecondOrder2024} to accelerate the Fruchterman--Reingold algorithm.

% explain the structure of the paper

The rest of the paper is organized as follows.
In Section~\ref{sec:preliminary}, we define an optimization problem for Fruchterman--Reingold algorithm.
In Section~\ref{sec:algorithm}, we introduce the Random Subspace method.
In Section~\ref{sec:experiment}, we show the experimental results.
Finally, we conclude the paper in Section~\ref{sec:conclusion}.


\section{Preliminary}\label{sec:preliminary}

In this section, we define an optimization problem for Fruchterman--Reingold algorithm.
Let $\bbR_+$ be a set of positive real numbers
and $A = (a_{i,j}) \in \bbR_+^{n \times n}$ be an adjacency matrix of a graph $G = (V, E)$.
Each vertex $v_i \in V$ is assigned a position $x_i \in \bbR^d$ and we define $x = (x_1, \dots, x_n) \in \bbR^{d \times n}$ as a matrix of positions.
For an optimal distance $k$, and a distance $d$ between two vertices $v_i$ and $v_j$, Fruchterman and Reingold defined the power of attraction $F_{i,j}^a: \bbR_+ \to \bbR$ and the power of repulsion $F^r: \bbR_+ \to \bbR$ as
\begin{equation*}
  F_{i,j}^a(d) \defeq \frac{a_{i,j} d^2}{k}, \quad F^r(d) \defeq -\frac{k^2}{d}.
\end{equation*}
The energy for these powers $E_{i,j}^a,E^r$ and the total energy $E_{i,j}$, stress of the graph, are defined as
\begin{align*}
  E_{i,j}^a(d) & \defeq \int_{0}^{d} F_{i,j}^a(r) \dd{r} = \frac{a_{i,j} d^3}{3k}, \\
  E^r(d)       & \defeq \int_{\infty}^{d} F^r(r) \dd{r} = -k^2\log{d},             \\
  E_{i,j}(d)   & \defeq
  \begin{cases}
    E_{i,j}^a(d) + E^r(d) & \text{if $i \neq j$}, \\
    0                     & \text{if $i = j$}.
  \end{cases}
\end{align*}
The energy function $E_{i,j}$ is convex for $a_{i,j} \in \bbR_+$ and minimized when $d = k\sqrt[3]{a_{i,j}}$.
Based on these energies, the optimization problem for Fruchterman--Reingold algorithm is defined as
\begin{mini}
  {x \in \bbR^{d \times n}}
  {f(x) \defeq \sum_{i,j} E_{i,j}(\norm{x_i - x_j})}
  {\label{eq:fr}}
  {}
\end{mini}

In order to introduce the Random Subspace method, we define a function $f_i: \bbR^{d} \to \bbR$ for $1 \leq i \leq n$ as
\begin{equation*}
  f_i(x_i) \defeq \sum_{1 \leq j \leq n} E_{i,j}(\norm{x_i - x_j}).
\end{equation*}
The gradient and the Hessian of $f_i$ are
\begin{align*}
  \nabla f_i(x_i)   & = \sum_{j \neq i} (x_i-x_j) \qty(\frac{a_{i,j}\norm{x_i-x_j}}{k} - \frac{k^2}{\norm{x_i - x_j}}),                    \\
  \nabla^2 f_i(x_i) & = \sum_{j \neq i} \qty(\frac{a_{i,j}\norm{x_i-x_j}}{k} - \frac{k^2}{\norm{x_i - x_j}}) I_d +                         \\
                    & \sum_{j \neq i} \qty(\frac{a_{i,j}}{k \norm{x_i-x_j}} + \frac{k^2}{\norm{x_i - x_j}^3}) (x_i - x_j)(x_i - x_j)^\top.
\end{align*}

As pointed out in~\cite{tunkelang1999numerical},
the Fruchterman--Reingold algorithm is a gradient descent method for the energy function $f$.


\section{Algorithm} \label{sec:algorithm}

\section{Experiment} \label{sec:experiment}

We used dataset from \cite{davis2011university} and MatrixMarket \cite{boisvertMatrixMarketWeb1997}.

\url{https://reference.wolfram.com/language/tutorial/GraphDrawingIntroduction.html}

\section{Conclusion} \label{sec:conclusion}

todo

\section*{Acknowledgment}

The author would like to thank ...

\ifthenelse{\boolean{isMain}}{
  \bibliographystyle{IEEEtran}
  \bibliography{FruchtermanReingoldByRandomSubspace}
}{}

\end{document}